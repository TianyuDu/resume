%%%%%%%%%%%%%%%%%%%%%%%%%%%%%%%%%%%%%%%%%
% Medium Length Professional CV
% LaTeX Template
% Version 2.0 (8/5/13)
%
% This template has been downloaded from:
% http://www.LaTeXTemplates.com
%
% Original author:
% Rishi Shah 
%
% Important note:
% This template requires the resume.cls file to be in the same directory as the
% .tex file. The resume.cls file provides the resume style used for structuring the
% document.
%
%%%%%%%%%%%%%%%%%%%%%%%%%%%%%%%%%%%%%%%%%

%----------------------------------------------------------------------------------------
%	PACKAGES AND OTHER DOCUMENT CONFIGURATIONS
%----------------------------------------------------------------------------------------

\documentclass{resume} % Use the custom resume.cls style

\usepackage[left=0.75in,top=0.6in,right=0.75in,bottom=0.6in]{geometry} % Document margins
\newcommand{\tab}[1]{\hspace{.2667\textwidth}\rlap{#1}}
\newcommand{\itab}[1]{\hspace{0em}\rlap{#1}}
\name{Tianyu Du} % Your name
\address{Undergraduate Student Studying Economics and Mathematics at University of Toronto}
%\address{www.tianyudu.com \\ www.github.com/tianyudu} % Your address
%%\address{123 Pleasant Lane \\ City, State 12345} % Your secondary addess (optional)
%\address{(+1) 647-886-7951 \\ tianyu.du@mail.utoronto.ca} % Your phone number and email

\begin{document}

\begin{rSection}{Contacts \& Personal Info}
	\textbf{Email} \texttt{tianyu.du@mail.utoronto.ca}
	\\
	\textbf{Phone} \texttt{(+1)647-886-7951}
	\\
	\textbf{Website} \texttt{www.tianyudu.com}
	\\
	\textbf{Github} \texttt{www.github.com/tianyudu}
	\\
	\textbf{LinkedIn} \texttt{https://www.linkedin.com/in/tianyu-du-7a56a7155}
\end{rSection}

%----------------------------------------------------------------------------------------
%	EDUCATION SECTION
%----------------------------------------------------------------------------------------

\begin{rSection}{Education}
% --------------------------------------------------------
{\bf University of Toronto, Canada} \hfill {\em Sep. 2017 - Jun. 2020(Expected)} 
\\ Honours Bachelor of Science, Economics \& Mathematics Specialist
%\\ Courses: Real Analysis, Game Theory, Non-linear Optimization. Time Series Analysis, Econometrics, Machine Learning, Neural Networks.
\\ Cumulative GPA: 4.00/4.00, Course Average: 94\%.
\\
% --------------------------------------------------------
\\{\bf Stanford University, United States} \hfill {\em Jun. 2019 - Aug. 2019} 
\\ Summer Session, Intensive Program in Data Science
\\ Courses: CS229:Machine learning, STATS202:Data Mining and Analysis, STATS116:Theory of Probability(Undergraduate).
\\ Cumulative GPA: 4.30/4.30, Course Average: 99\%.
%\\
%% --------------------------------------------------------
%\\{\bf Hangzhou Foreign Languages School, China} \hfill {\em Sep. 2014 - Jun. 2017} 
%\\ Examinations: General Certificate of Education A-Level(CIE). Advanced Placement(AP).
%\\ Activities: Co-founder of HwHumans Student Platform.
\end{rSection}

\begin{rSection}{Research Interests}
Machine Learning Methods and their Applications on Time Series Forecasting.
\\Computational Economics, Simulations for Game Theory, and Market Design.
\\Behavioural and Experimental Economics.
\end{rSection}


\begin{rSection}{Scholarships \& Awards}
{Alexander Mackenzie Scholarship In Economics And Political Science} \hfill {\em Oct. 2019} \\
{Dean's List Scholar(2018-19)} \hfill {\em Jun. 2019} \\
{International Experience Award (Killam American Fund for International Exchange)} \hfill{\emph May. 2019} \\
{Dean's List Scholar(2017-18)} \hfill {\em Jan. 2018}
\end{rSection}

%--------------------------------------------------------------------------------
%    Projects And Seminars
%-----------------------------------------------------------------------------------------------
\begin{rSection}{Activities \& Projects}
{\bf Thesis on Stock Market Forecasting from Textual Data}
\\ \emph{Honours Essay in Applied Microeconomics} \hfill \emph{Sep. 2019 - Apr. 2020}
\\ Top students from department of economics are selected to conduct their own original research in this program. My thesis focuses on forecast asset market movements from financial news. Specifically, sentimental analysis tools from natural language processing (NLP) are used to generate article-level sentiment scores. Then data science techniques including SVM, CNN-RNN are deployed to create predictive models and capture the underlying inter-temporal dependencies.
\\
\\
{\bf Patient Data Analysis on PANSS Dataset} \hfill \emph{Jun.2019 - Aug.2019}
\\ \emph{The Final Project for STATS202 at Stanford University (Final Report Class Top)}
\\ Positive and Negative Syndrome Scale (PANSS) scores of schizophrenia patients were used to test treatment effects, k-means and Gaussian mixture were used to cluster patients based on scores prior to treatment. Moreover, SVM, random forests, and boosting machines were developed to detect potential invalid assessments and forecast patients' future psychological states.
\\
\\{\bf Artificial Neural Networks for Economic Forecasting} \hfill \emph{May.2018 - Jun.2019}
\\ \emph{Independent Research}
\\ This project compared artificial neural networks and classical models on financial time series. Specifically, fully connected and RNN with LSTM cells were used on exchange rate forecasting, which had outperformed existing ARIMA and VAR models.
\\
\\{\bf Independent Reading in Mathematics: Mathematical Economics} \hfill \emph{May.2019 - Jun.2019}
\\ \emph{Supervisor: Robert J. McCann}
\\
A supervised learning program focusing on microeconomic theory with mathematical rigour. Topics included duality theory in optimization, consumer and producer theory, partial and general equilibrium, as well as market failures like adverse selection.
\\
\\{\bf CIBC Machine Intelligence Hackathon} \hfill \emph{Sep.2018}
\\ \emph{Finalist Group (Top 5)}
\\
An auto-encoder-decoder architecture neural network was implemented to detect fraud in medical insurance claims.
\\

\end{rSection}

%--------------------------------------------------------------------------------
%    Skills
%-----------------------------------------------------------------------------------------------
\begin{rSection}{Skills}
\textbf{Programmings} Python including TensorFlow, PyTorch, Sci-kit Learn, Pandas, Numpy, and various data visualization toolkits; R; STATA; Matlab; Mathematica; Bash.
\\\textbf{Development} Server deployment on Amazon Web Services (AWS) and Google Cloud Platform (GCP).
\\\textbf{Data Analytics \& Machine Learning} Solid mathematical and statistical foundations for statistical learning models. Being able to implement and deploy machine learning models for both academic purposes such as paper replication and industrial purposes.
\end{rSection}

\begin{rSection}{Extra-Cirrucular} 
	Economics Peer Mentorship Program (as Mentor) \hfill \emph{Oct.2019 - Apr.2020}
\end{rSection}

\begin{rSection}{Other Courses \& Certificates}
	\textbf{Coursera} Practical Time Series Analysis; Machine Learning; Serverless machine learning with Tensorflow on Google cloud platform; Social and economic networks: models and analysis; Sequence models (recurrent neural networks); Mathematics for machine learning: multivariate calculus.
	\\
	\textbf{Nvidia} Accelerated computing with CUDA python.
\end{rSection}

%----------------------------------------------------------------------------------------
%	WORK EXPERIENCE SECTION
%----------------------------------------------------------------------------------------

%\begin{rSection}{Work Experience}
%
%\begin{rSubsection}{SJ Contracts, Pune}{June 2016}{Site Engineer}{}
%\item On-site internship under this leading construction company. Learned and implemented various aspects such as quantity estimation, labour management and safety precautions.
%\end{rSubsection}
%
%
%\end{rSection}


%	EXAMPLE SECTION
%----------------------------------------------------------------------------------------

%\begin{rSection}{Academic Achievements} 
% Runners up in B.G.Shirke Vidyarthi Competition for Innovative Project organized by Pune Construction Engineering Research Foundation in January 2018
%\item Won First Prize in Model Making Competition Organized by Symbiosis Institute of Technology, Pune.
%\end{rSection}
%
%\newpage

%----------------------------------------------------------------------------------------
% Extra Curricular
%----------------------------------------------------------------------------------------
%\begin{rSection}{Extra-Cirrucular} \itemsep -3pt
%\item Co-Organized “ Nirmitee 2017” - a National Symposium of Civil Department of MIT, Pune
%\item Attended a workshop on Autodesk Revit at IIT Bombay in 2014.
%\item Winner of Inter Departmental Football Competition 2015.
%\item Member of the  Rotaract Club Of Pune Pride from 2014 to 2017.
%\item Worked for a start-up company Named OUST as a Regional Marketing Manager
%\item Trained and disciplined in National Cadet Corps (NCC), IIT Kanpur for a year.
 %\item  Participated in Vijyoshi Camp 2012 organized at Indian Institute of Science, Bangalore.
 %\item Won 2nd position in Kho-Kho in Intramurals conducted by Physical Education Section, IIT Kanpur.
 %\item Pursued French as second language during secondary school from Grade 6 to Grade 10. Also participated in French Song Competition and French G.K. Quiz in Class 10th. %

%\end{rSection}
%
%\begin{rSection}{Personal Traits}
%\item Highly motivated and eager to learn new things.
%\item Strong motivational and leadership skills.
%\item Ability to work as an individual as well as in group.
%\end{rSection}
\end{document}
