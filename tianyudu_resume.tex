%%%%%%%%%%%%%%%%%%%%%%%%%%%%%%%%%%%%%%%%%
% Medium Length Professional CV
% LaTeX Template
% Version 2.0 (8/5/13)
%
% This template has been downloaded from:
% http://www.LaTeXTemplates.com
%
% Original author:
% Rishi Shah 
%
% Important note:
% This template requires the resume.cls file to be in the same directory as the
% .tex file. The resume.cls file provides the resume style used for structuring the
% document.
%
%%%%%%%%%%%%%%%%%%%%%%%%%%%%%%%%%%%%%%%%%

%----------------------------------------------------------------------------------------
%	PACKAGES AND OTHER DOCUMENT CONFIGURATIONS
%----------------------------------------------------------------------------------------

\documentclass{resume} % Use the custom resume.cls style

\usepackage{datetime}
\usepackage[left=0.75in,top=0.6in,right=0.75in,bottom=0.6in]{geometry} % Document margins
\newcommand{\tab}[1]{\hspace{.2667\textwidth}\rlap{#1}}
\newcommand{\itab}[1]{\hspace{0em}\rlap{#1}}
\name{Tianyu Du} % Your name
\address{Undergraduate at University of Toronto, Economics and Mathematics Specialist Program}
%\address{www.tianyudu.com \\ www.github.com/tianyudu} % Your address
%%\address{123 Pleasant Lane \\ City, State 12345} % Your secondary addess (optional)
%\address{(+1) 647-886-7951 \\ tianyu.du@mail.utoronto.ca} % Your phone number and email

\begin{document}
\footnotetext{This copy is accurate up to \currenttime\ \today}

\begin{rSection}{Contacts}
	\textbf{Email} \texttt{tianyu.du@mail.utoronto.ca}
	\quad
	\textbf{Phone} \texttt{(+1)647-886-7951}
	\quad
	\textbf{Website} \texttt{www.tianyudu.com}
	\quad
	\textbf{Github} \texttt{www.github.com/tianyudu}
	\quad \quad 
	\textbf{LinkedIn} \texttt{www.linkedin.com/in/tianyu-du}
\end{rSection}

%----------------------------------------------------------------------------------------
%	EDUCATION SECTION
%----------------------------------------------------------------------------------------

\begin{rSection}{Education}

{\bf Stanford University} \hfill {\em September 2020 - June 2022} 
\\ \emph{Master of Science in Management Science \& Engineering}
\\ Incoming Graduate Student, Focus: Computational Social Science.
\\
% --------------------------------------------------------
\\ {\bf University of Toronto} \hfill {\em September 2017 - June 2020} 
\\ \emph{Honours Bachelor of Science, Economics \& Mathematics}
%\\ Courses: Calculus, Linear Algebra,Real Analysis, Game Theory, Non-linear Optimization. Time Series Analysis, Econometrics, Probability, Machine Learning, Neural Networks.
\\ GPA: 4.00/4.00, Course Average: 95\%.
\\ Thesis: Efficiency of the Crude Oil Market and Forecasting Crude Oil Returns using News Sentiments
\\ Supervisors: Stuart M. Turnbull and Aloysius Siow
\\
% --------------------------------------------------------
\\{\bf Stanford University} \hfill {\em June 2019 - August 2019} 
\\ \emph{Summer Session, Program of Intensive Studies in Data Science}
%\\ Courses: CS229:Machine Learning(Graduate), STATS202:Data Mining and Analysis(Graduate), STATS116: Theory of Probability(Undergraduate).
\\ GPA: 4.30/4.30, Course Average: 99\%.
\\ Project: Patient Data Analysis on PANSS Dataset \emph{($1^{st}$ place in class)}
\\ Instructor: Linh Tran
% --------------------------------------------------------
%\\{\bf Hangzhou Foreign Languages School, China} \hfill {\em September 2014 - June 2017} 
%\\ Examinations: General Certificate of Education A-Level(CIE). Advanced Placement(AP).
%\\ Activities: Co-founder of HwHumans Student Platform.
\end{rSection}

\begin{rSection}{Research Interests}
Machine Learning Methods and their Applications on Time Series Forecasting.
\\Computational Social Sciences.
\end{rSection}


\begin{rSection}{Scholarships \& Awards}
{Mcnab Undergarduate In-Course Scholarship} \hfill {\em December 2019} \\
{Alexander Mackenzie Scholarship In Economics And Political Science} \hfill {\em October 2019} \\
{Killam American Fund For International Exchange} \hfill{\emph May 2019} \\
{Dean's List Scholar2017-18 and 2018-19} \hfill {\em 2017-2018}
\end{rSection}

%--------------------------------------------------------------------------------
%    Projects And Seminars
%-----------------------------------------------------------------------------------------------
\begin{rSection}{Activities \& Projects}
%{\bf Thesis on Forecasting Crude Oil Returns using News Sentiment and Machine Learning}
%\\ \emph{Honours Essay in Applied Microeconomics} \hfill \emph{September 2019 - April 2020}
%\\ Top students from department of economics are selected to conduct their own original research in this program. My thesis focuses on forecasting spot price of crude oil using news sentiments. Dataset from Ravenpack news analytics are used to construct daily sentimental measures for crude oil market. Data science techniques including SVM, Random Forest, LSTM, CNN-RNN are deployed to create predictive models and capture the underlying inter-temporal dependencies.
%\\
%\\
{\bf TD Rotman FinHub TDMDAL Hackathon} \emph{Finalist Group (Top 5)} \hfill \emph{February 2020}
\\ In this project, we developed a dictionary based NLP process extracting information from transcripts of earning calls of S\&P 500 companies, and predict stock price movement on the next trading day.
%\\
%\\{\bf Patient Data Analysis on PANSS Dataset} \emph{($1^{st}$ place in class)} \hfill \emph{June 2019 - August 2019}
%\\ \emph{The Final Project for STATS202 at Stanford University}
%\\ Positive and Negative Syndrome Scale (PANSS) scores of schizophrenia patients were used to test treatment effects, k-means and Gaussian mixture were used to cluster patients based on scores prior to treatment. SVM, random forests, and boosting machines were developed to detect potential invalid assessments and forecast patients' future psychological states.
%\\
%\\{\bf Artificial Neural Networks for Economic Forecasting} \hfill \emph{May.2018 - Jun.2019}
%\\ \emph{Independent Research}
%\\ This project compared artificial neural networks and classical models on financial time series. Specifically, fully connected and RNN with LSTM cells were used on exchange rate forecasting, which had outperformed existing ARIMA and VAR models.
%\\
%\\{\bf Special Topics in Mathematics: Mathematical Economics} \hfill \emph{May 2019 - June 2019}
%\\ \emph{Supervisor: Robert J. McCann}
%\\
%A supervised learning program focusing on microeconomic theory with mathematical rigour. Topics included duality theory in optimization, consumer and producer theory, partial and general equilibrium, as well as market failures like adverse selection.
%\\
\\{\bf CIBC Machine Intelligence Hackathon} \emph{Finalist Group (Top 5)} \hfill \emph{September 2018}
\\
An auto-encoder-decoder architecture neural network was implemented to detect fraud in medical insurance claims.
\\

\end{rSection}

%--------------------------------------------------------------------------------
%    Skills
%-----------------------------------------------------------------------------------------------
\begin{rSection}{Skills}
\textbf{Programming Languages and Libraries} Python, PyTorch, Sci-kit Learn, Pandas, Numpy, Matplotlib, R, STATA, Matlab, Mathematica, Bash, Latex.
\\\textbf{Development} Git, Server deployment on Amazon Web Services (AWS) and Google Cloud Platform (GCP).
\end{rSection}

%\begin{rSection}{Recent Extra-Cirrucular} 
%	Volunteer: Economics Peer Mentorship Program, as Mentor. \hfill \emph{October 2019 - April 2020}
%	\\
%	Volunteer: University of Toronto, Representative at the Learning Abroad Fair. \hfill \emph{November 2019}
%	\\
%	Volunteer: University of Toronto, Second Year Learning Community Panel, as Panelist. \hfill \emph{October 2019}
%\end{rSection}

%\begin{rSection}{Other Courses \& Certificates}
%	\textbf{Coursera} Practical Time Series Analysis; Machine Learning; Serverless machine learning with Tensorflow on Google cloud platform; Social and economic networks: models and analysis; Sequence models (recurrent neural networks); Mathematics for machine learning: multivariate calculus.
%	\\
%	\textbf{Nvidia} Accelerated computing with CUDA python.
%\end{rSection}

%----------------------------------------------------------------------------------------
%	WORK EXPERIENCE SECTION
%----------------------------------------------------------------------------------------

%\begin{rSection}{Work Experience}
%
%\begin{rSubsection}{SJ Contracts, Pune}{June 2016}{Site Engineer}{}
%\item On-site internship under this leading construction company. Learned and implemented various aspects such as quantity estimation, labour management and safety precautions.
%\end{rSubsection}
%
%
%\end{rSection}


%	EXAMPLE SECTION
%----------------------------------------------------------------------------------------

%\begin{rSection}{Academic Achievements} 
% Runners up in B.G.Shirke Vidyarthi Competition for Innovative Project organized by Pune Construction Engineering Research Foundation in January 2018
%\item Won First Prize in Model Making Competition Organized by Symbiosis Institute of Technology, Pune.
%\end{rSection}
%
%\newpage

%----------------------------------------------------------------------------------------
% Extra Curricular
%----------------------------------------------------------------------------------------
%\begin{rSection}{Extra-Cirrucular} \itemsep -3pt
%\item Co-Organized “ Nirmitee 2017” - a National Symposium of Civil Department of MIT, Pune
%\item Attended a workshop on Autodesk Revit at IIT Bombay in 2014.
%\item Winner of Inter Departmental Football Competition 2015.
%\item Member of the  Rotaract Club Of Pune Pride from 2014 to 2017.
%\item Worked for a start-up company Named OUST as a Regional Marketing Manager
%\item Trained and disciplined in National Cadet Corps (NCC), IIT Kanpur for a year.
 %\item  Participated in Vijyoshi Camp 2012 organized at Indian Institute of Science, Bangalore.
 %\item Won 2nd position in Kho-Kho in Intramurals conducted by Physical Education Section, IIT Kanpur.
 %\item Pursued French as second language during secondary school from Grade 6 to Grade 10. Also participated in French Song Competition and French G.K. Quiz in Class 10th. %

%\end{rSection}
%
%\begin{rSection}{Personal Traits}
%\item Highly motivated and eager to learn new things.
%\item Strong motivational and leadership skills.
%\item Ability to work as an individual as well as in group.
%\end{rSection}
\end{document}
